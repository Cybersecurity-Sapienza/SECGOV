\documentclass[12pt,a4paper,oneside]{article}
\usepackage{graphicx}
\usepackage{titlepic}
\usepackage[utf8]{inputenc}
\usepackage[left=1.1in,right=1.1in, top=1in, bottom= 1in]{geometry}
\usepackage{amsfonts}
\usepackage{amssymb}
\usepackage{amsmath}
\usepackage{fancyhdr}
\usepackage{hyperref}
\usepackage{etoolbox}
\usepackage[nottoc]{tocbibind}
\usepackage{appendix}
\usepackage{multicol}
\usepackage{leftidx}
%\graphicspath{{figures/}}
\usepackage{ragged2e}
\usepackage{mathtools}
\usepackage{units}
\usepackage{float}
\usepackage{subcaption}
\usepackage{commath}
\usepackage{xcolor}
\usepackage{listings}
\usepackage{amsthm}
\usepackage{censor}
\usepackage{framed}

\usepackage[none]{hyphenat} % Avoids to go out of margin

\usepackage{subfiles}

% --------------------------------------------- %

\title{FAEQ}
\author{Riccardo Versetti}
\date{\today}
\usepackage{caption}
\captionsetup[figure]{font=small}
%\captionsetup[table]{font=small}

%\usepackage{setspace} % double spacing
\linespread{1.2}

\makeatletter
\let\thetitle\@title
\let\theauthor\@author
\let\thedate\@date
\makeatother

% Aggiunta del comando "quiz"
\newcounter{quizcounter}
\newcommand{\quiz}[2]{%
    \refstepcounter{quizcounter} % Incrementa il contatore
    \textcolor{red}{
        \noindent\textbf{\thequizcounter)} \textbf{#1}
    }
    \\
    %\textbf{Risposta:} 
    #2\\[1em]
}

\begin{document}

\lstset{
    language=bash,
    basicstyle=\ttfamily,
    keywordstyle=\color{blue},
    stringstyle=\color{green},
    showstringspaces=false
}

\include{title}
\newpage
\tableofcontents
\pagenumbering{roman}
\newpage

\pagenumbering{arabic}
\section{ISG model}

\quiz
{
    Descrivi il modello ISG di Von Solms e discuti l’importanza di implementare i processi di Direzione, Esecuzione e Controllo come un ciclo.
}
{
    Il modello ISG (Information Security Governance) di Von Solms è un framework per la gestione della sicurezza delle informazioni, che si basa su tre processi principali: Direzione, Esecuzione e Controllo. Questo modello enfatizza l’importanza di un approccio sistematico per garantire che la sicurezza delle informazioni sia gestita in modo continuo e coerente con gli obiettivi aziendali. 

    \begin{itemize}
        \item La \textbf{direzione} stabilisce le politiche, le strategie e gli obiettivi per la sicurezza delle informazioni. Questa fase è cruciale per fornire una guida chiara su come la sicurezza dovrebbe essere integrata nei processi aziendali.
        \item L'\textbf{esecuzione} si occupa della realizzazione delle politiche e delle strategie definite nella fase di Direzione. Coinvolge la gestione operativa della sicurezza, come l’implementazione di controlli tecnici, organizzativi e procedurali.
        \item Il \textbf{controllo} valuta l’efficacia delle misure adottate attraverso audit, monitoraggio continuo e feedback. Permette di identificare eventuali lacune o miglioramenti necessari.
    \end{itemize}

    L’implementazione di questi processi come un ciclo continuo è fondamentale per garantire un adattamento costante alle nuove minacce e vulnerabilità che possono emergere. Inoltre, questo approccio favorisce il miglioramento continuo, grazie all’analisi dei risultati ottenuti e al feedback raccolto lungo il percorso. Infine, permette di mantenere un allineamento costante tra gli obiettivi di sicurezza e quelli aziendali, assicurando che la gestione della sicurezza delle informazioni rimanga coerente con le esigenze di un contesto di business in continua evoluzione.
}

\quiz{
    Descrivi, con riferimento al modello ISG di Von Solms, la dimensione "depth" e spiega come si relaziona al ciclo DEC.
}
{
    Risposta
}

\end{document}